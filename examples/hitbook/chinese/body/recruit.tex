% !Mode:: "TeX:UTF-8"

\chapter{哈尔滨工业大学初砚硕课题组招(聘)全日制硕士、博士、助理研究员}[Yanshuo Chu lab is recruiting full-time Master’s and PhD students, as well as Assistant Research Fellows.]

\section{导师简介}[Principal Investigator (PI) Introduction]

教授,博士生导师,国家高层次青年人才。2019年于哈尔滨工业大学获得计算机应用技术博士学位,先后以博士后、数据科学家身份任职于美国MD安德森癌症研究中心基因医学系。2025年3月任职于哈尔滨工业大学生命科学和医学学部(生物医学工程学科)和计算学部(智能科学与技术学科),两个学科都招收硕博士。

\section{主要研究方向}[Main Research Directions]

肿瘤信息学,肿瘤进化重构,肿瘤数字孪生

\section{学术成果}[Academic Achievements]

近五年以第一作者在Nature Medicine(2023)、Nature Communications(2022)发表论文2篇,以合作作者在Nature、Cancer Cell等业内顶级期刊发表论文10篇。
其中Nature Medicine(2023)被Nature杂志特别报道,入选ESI高被引论文,荣获美国MD安德森癌症中心2023财年杰出科学奖。

\section{团队介绍}[Research Team Introduction]

团队经费充沛、基础雄厚。所在团队于2019年申请并获批“生物大数据”教育部重点实验室,建立了国内先进水平的基因组大数据专用高性能计算集群,拥有流式细胞仪、激光共聚焦显微镜等国际一流水平的仪器设备,能够开展生物化学、分子生物学、细胞生物学、基因组学、生物信息学等各项实验。

\section{其他介绍}[Other Information]

强AI背景下,干实验室将走向无人化。目前我正基于哈尔滨医科大学医工交叉学院的计算集群开发和部署自动化实验室管理系统,目标是开发首个无人化实验室。目前,该平台已能够\textbf{超快速}、\textbf{大规模}进行单细胞、bulk、空间测序数据分析,\textbf{极大缩短研究周期}、\textbf{快速得到研究成果}。而在个人PC端,推荐使用我开发的GPT2Org+elfeed工具链:

- GPT2Org,一款
\href{https://chromewebstore.google.com/detail/gpt2org/nedljjoclmlgpobfohcnanjipnblgjbo?pli=1}{\uline{Chrome}}、
\href{https://addons.mozilla.org/en-US/firefox/addon/gpt2org/}{\uline{Firefox}}浏览器插件,使用\href{https://platform.openai.com/api-keys}{\uline{OpenAI}}或\href{https://platform.deepseek.com/api_keys}{\uline{DeepSeek}}的API,一键总结网页内容(学术论文),并能够使用org-protocol自动发送到Emacs中,可以无缝整合org-roam知识库。
- ElfeedAI,智能文献管理工具,可以自动更新论文并对论文重要性排序并进行AI阅读并总结。我们不再需要去搜索文章,也不再需要详细阅读文章,这将极大节省注意力和时间,因为“Attention Is All You Need”。
\href{https://www.bilibili.com/video/BV1dModYPE3c/?share_source=copy_web&vd_source=d6f2f3603fdf3b24c20399fbfd25df0d}{\uline{点击这里:B站的演示视频【如何利用AI搞科研?】}}

以上,\textbf{通过在计算集群端+个人PC端深度融合AI,将会有无限可能}。

\section{硕博士招生要求}[Graduate Student Recruitment Requirements]

\begin{itemize}
\item 计算机、数学、生命相关专业,熟悉常用编程语言。
\item 对科研和学术有一定的追求,有进取心,踏实肯干。
\item 发表过相关的高水平论文或有大规模软件、数学建模项目经验,优先。
\end{itemize}

\section{助理研究员招聘需求和待遇}[Assistant Researcher Position]

招聘\textbf{数学、统计或计算机}相关方向‌,工作地点在哈工大本部,招聘人数‌1-2人。负责:
软件开发与算法实现‌、参与科研项目的软件系统开发,完成算法设计、代码实现及性能优化。
维护和升级现有科研工具或仿真平台,确保其稳定性和扩展性。
针对复杂问题建立数学模型,进行理论分析或数值模拟。
跨学科协作‌,与团队其他成员(如生命科学研究人员)合作,推动研究成果落地。
\textbf{技能要求}:熟练掌握Python/C++/MATLAB等至少一门语言,有软件开发经验。
扎实的数学基础(如优化理论、微分方程、概率统计等),具备数学建模经验。
发表过SCI/SSCI论文者优先,或具备较强的中英文技术文档撰写能力。
\textbf{福利待遇}:底薪25万/年+绩效。提供国内外学术会议交流机会与职业发展支持。

\section{申请方式}[Application Method]

发送邮件:yanshuochu@hit.edu.cn

% Local Variables:
% TeX-master: "../thesis"
% TeX-engine: xetex
% End:
